\documentclass{article}
\usepackage[utf8]{inputenc}
\usepackage[T1]{fontenc}
\usepackage{hyperref}
\usepackage{graphicx}
\usepackage[normalem]{ulem}
\usepackage{listings}
\usepackage{booktabs}
\usepackage{xcolor}
\begin{document}
\section{Listy}



\subsection{Lista nieuporządkowana}



\begin{itemize}
  \item Pierwszy element

  \item Drugi element

  \item Trzeci element
\begin{itemize}
  \item Zagnieżdżony element

  \item Kolejny zagnieżdżony element

\end{itemize}


  \item Czwarty element

\end{itemize}



\subsection{Lista uporządkowana}



\begin{enumerate}
  \item Pierwszy krok

  \item Drugi krok

  \item Trzeci krok
\begin{enumerate}
  \item Zagnieżdżony krok

  \item Kolejny zagnieżdżony krok

\end{enumerate}


  \item Czwarty krok

\end{enumerate}



\subsection{Lista zadań}



\begin{enumerate}
  \item[$\square$] Zadanie do wykonania

  \item[$\boxtimes$] Zadanie wykonane

  \item[$\square$] Kolejne zadanie

  \item[$\boxtimes$] Inne wykonane zadanie

  \item[$\square$] Zadanie z \textbf{pogrubionym} tekstem

  \item[$\boxtimes$] Zadanie z \textit{pochylonym} tekstem

\end{enumerate}



\subsection{Mieszane listy}



\begin{enumerate}
  \item Pierwszy element listy numerowanej
\begin{itemize}
  \item Zagnieżdżony element punktowany

  \item Kolejny zagnieżdżony element

\end{itemize}


  \item Drugi element listy numerowanej
\begin{enumerate}
  \item[$\square$] Zagnieżdżone zadanie

  \item[$\boxtimes$] Wykonane zagnieżdżone zadanie

\end{enumerate}


\end{enumerate}


\end{document}
