\documentclass{article}
\usepackage[utf8]{inputenc}
\usepackage[T1]{fontenc}
\usepackage{hyperref}
\usepackage{graphicx}
\usepackage[normalem]{ulem}
\usepackage{listings}
\usepackage{booktabs}
\usepackage{xcolor}
\begin{document}
\section{Tabele}



\subsection{Podstawowa tabela}



\begin{tabular}{|l|l|l|}
\toprule
\textbf{Nagłówek 1} & \textbf{Nagłówek 2} & \textbf{Nagłówek 3} \\
\midrule
Komórka 1 & Komórka 2 & Komórka 3 \\
Komórka 4 & Komórka 5 & Komórka 6 \\
\bottomrule
\end{tabular}



\subsection{Tabela z wyrównaniem}



\begin{tabular}{|l|l|l|}
\toprule
\textbf{Lewo} & \textbf{Środek} & \textbf{Prawo} \\
\midrule
L1 & Ś1 & P1 \\
L2 & Ś2 & P2 \\
L3 & Ś3 & P3 \\
\bottomrule
\end{tabular}



\subsection{Tabela z formatowaniem}



\begin{tabular}{|l|l|l|}
\toprule
\textbf{Nazwa} & \textbf{Opis} & \textbf{Status} \\
\midrule
\textbf{Projekt A} & \textit{Ważny projekt} & Gotowy \\
\textbf{Projekt B} & \sout{Anulowany} & Anulowany \\
\textbf{Projekt C} & \texttt{W trakcie} & W trakcie \\
\bottomrule
\end{tabular}



\subsection{Tabela z linkami}



\begin{tabular}{|l|l|l|}
\toprule
\textbf{Strona} & \textbf{URL} & \textbf{Opis} \\
\midrule
Google & \href{https://google.com}{google.com} & Wyszukiwarka \\
GitHub & \href{https://github.com}{github.com} & Hosting kodu \\
Stack Overflow & \href{https://stackoverflow.com}{stackoverflow.com} & Q\&A dla programistów \\
\bottomrule
\end{tabular}


\end{document}
