\documentclass{article}
\usepackage[utf8]{inputenc}
\usepackage[T1]{fontenc}
\usepackage{hyperref}
\usepackage{graphicx}
\usepackage[normalem]{ulem}
\usepackage{listings}
\usepackage{booktabs}
\usepackage{xcolor}
\begin{document}
\section{Cytaty i podziały}



\subsection{Prosty cytat}



\begin{quote}
To jest prosty cytat.
Może zawierać wiele linii.
\end{quote}



\subsection{Cytat zagnieżdżony}



\begin{quote}
To jest główny cytat.
\begin{quote}
To jest cytat zagnieżdżony.
\begin{quote}
To jest cytat podwójnie zagnieżdżony.
\end{quote}

\end{quote}

\end{quote}



\subsection{Cytat z formatowaniem}



\begin{quote}
\textbf{Ważny cytat} z pogrubionym tekstem.

\textit{Pochylony tekst} w cytacie.

- Lista w cytacie
- Kolejny element listy
\end{quote}



\subsection{Łamanie linii}



To jest pierwsza linia.  
To jest druga linia po łamaniu.




To jest nowy akapit po pustej linii.




\subsection{Linia pozioma}



Tekst przed linią poziomą.




\noindent\rule{\linewidth}{0.4pt}



Tekst po linii poziomej.




\noindent\rule{\linewidth}{0.4pt}



Kolejna linia pozioma.




\noindent\rule{\linewidth}{0.4pt}



Ostatnia linia pozioma.



\end{document}
